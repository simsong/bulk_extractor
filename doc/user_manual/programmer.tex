Appendix


Writing \bulk scanners.

Design

\bulk scanners are written in C++ but are called with ``C'' linkage
for simplicity. 

Phases

Your scanner will be called in phases. You should check for each
phase, rather than assuming that these are the only phases (because
new phases may be added).

Phase 0 - startup/loading
phase 1 - scans a sbuf
phase 2 - shutdown
phase 3 - post-initialization startup.


Packaging

Your scanner should ideally consist of a single .cpp file
(\emph{scanner.cpp}). There really is no need for a .h file. If your
scanner will be linked in to the \bulk then you should put your
scanner's name at the end of the \emph{bulk\_extractor.h} file. If
not, then don't.

Unicode: If you have need for unicode support (and who doesn't?), then you
should \verb|#include "utf8.h"| to use the excellent GNU UTF-8 support package.



Efficiency




Multi-threading

\bulk's architecture allows you to write code that executes in a
multi-threaded environment with little effort on your part. You do not
need to use mutexes, spin-locks, pthreads, synchronization elements,
or any other traditional threading mechanisms in your code. Instead,
\bulk handles this all for you as part of the workload dispatch
system.

You must be aware, however, that your code \emph{is} running in a
multi-threaded environment. This means that multiple copies of your
code may be running at the same time and accessing the same global
variables. So you need to be careful! Many programming practices that
are acceptable in a single-threaded environment will generate
unexpected crashes in a multi-threaded environment. The good news is
that these are generally not very good programming practices, so you
shouldn't be using them only.

In general, here's what happens from a threading perspective when your
scanner runs, and how you should respond:


Here are a few simple rules to follow for each of the phases.

Phase 1 - Your scanner will be called in Phase 1 precisely
once. During this phase your scanner should initialize any global
variables that it needs to access. These variables should be
\emph{static} so that there is no possibility that they will impact
other scanners.

Phase 2 - Your scanner will be called in Phase 2 for every page that
is processed. Your scanner may also be called by multiple threads
simultaneously. The chance that multiple
threads will be in your scanner simultaneously increasingly linearly
with the amount of CPU load that your scanner creates, for the simple
reason that your scanner is consuming a larger fraction of all the
scanner time. 

You can test your scanner by running it as the \emph{only}
scanner. You can do this with the \emph{-E scanner} option, which
enables \emph{scanner} and disables all of the others.

Your scanner will be called a LOT. For processing a 1TB drive image,
your scanner will be called at least $1,000,000,000,000 / 16,777,216 =
59,604$ times. But remember that your scanner will also be called
every time \bulk recursively processes a block of data. In a typical
run against a 1TB drive image, assume that your scanner will be called
at least 500,000 times on data segments ranging from 16MiB to 4096
bytes in size.

In general, your scanner should keep all of its per-thread state on
the stack. Any global variable that your scanner accesses \emph{should
only be accessed with const pointers}. I cannot stress this
enough. Be sure to read these referneces:

\begin{itemize}
\item \emph{Const correctness}, part of the \emph{C++ FAQ}, which you can find at
   \url{http://www.parashift.com/c++-faq-lite/const-correctness.html}, 
\item \url{http://en.wikipedia.org/wiki/Const-correctness}
\item \url{http://en.wikipedia.org/wiki/Constant_(programming)}
\item \url{http://www.cprogramming.com/tutorial/const_correctness.html}
\item
  \url{http://www.cprogramming.com/reference/pointers/const_pointers.html}
\end{itemize}

All items that your scanner saves should be saved using \bulk's
feature recorder system. All of the exposed virtual methods methods
are guaranteed to be thread-safe.

If you must increment a global variable, use an
__sync_fetch_and_add. For example, this increments 1 to counter and
executes the function bar() if the value is equal to 10:

   if(__sync_fetch_and_add(&counter,1)==10){
     bar();
   }

If you want to test the value of counter without incrementing it, use
this:

   if(__sync_fetch_and_add(&counter,0)==10){
     bar();
   }

That is, \emph{once you start modifying a global variable in a
  multi-threaded environment, you may only access that variable using
  thread-aware code.} That's because the thread-aware code will
 use the correct barriers, mutexes or locks necessary on your system's
 architecture to get a current copy of the global variable. This is
 especially important on systems with multiple processes and multiple caches.

__sync_fetch_and_add is a GCC built-in. Find more information about it at
http://www.cprogramming.com/reference/pointers/const_pointers.html. 


Things you should not do:

Do not make a copy of a complex global data structure for your scanner
in phase 2. The overhead of making the copy will slow down  your
scanner. Instead, access the data structure read-only, and keep
whatever local state you need on the stack. Although this may require
redesigning third-party code, in general such changes are only
necessary on third-party code that is poorly designed.

Do not use library calls that are not thread safe. For example, do not
use localtime(), because it uses global state that is kept in the C
library. Instead use localtime_r().

Do not create a so-called Giant Lock for your scanner or an expensive library
call. A ``Giant Lock'' (\url{http://en.wikipedia.org/wiki/Giant_lock})
is a single lock that is used to protect a piece of code that is
hopelessly not thread safe. The Unix kernel used to have a single
Giant Lock because the kernel was developed for a single-processor
system and the developers assumed that only a single thread would be
running when the processor was in the kernel and interrupts were
turned off. When Unix was put on multi-processor machines the Giant
Lock prevented more than one processor from entering the kernel at a
time. Although this prevented the system from crashing, it imposed
unacceptable performance penalties, because it turned a
multi-processor system into a uni-processor whenever more than one
processor tried to enter the kernel at the same time.  Python also has
a giant lock, which it calls the Global Interpreter Lock
(\url{http://wiki.python.org/moin/GlobalInterpreterLock}) which is a
problem because it prevents multi-threaded Python programs from having
two threads execute Python bytecode at the same time. Python has the
GIL because the memory management code is not thread safe.

If you create a Giant Lock, then only one copy of your scanner will
run at a time. This is especially a problem if your scanner is
CPU-intensive: it will kill the overall performance  of \bulk and
users will respond by disabling your scanner. They will not be able to
make your scanner run faster by running it on faster hardware. 

Phase 3 - Only one instance of your scanner will run in Phase 3. Phase
3 does not run until all scanners have terminated. Thus, you may once
again modify global variables without the need to lock them.  However,
any variable that was modified in one of your threads must still be
accessed using synchronization primitives because of cache coherency
issues.







